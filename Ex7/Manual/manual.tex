\documentclass[a4paper,12pt]{article}

\usepackage[utf8]{inputenc}
\usepackage[catalan]{babel}
\usepackage{hyperref}
\usepackage{enumitem}
\usepackage{geometry}
\geometry{margin=1in}

\begin{document}

\title{Manual d'Usuari}
\author{Nom de l'Autor}
\date{\today}

\maketitle

\tableofcontents
\newpage

\section{Introducció}
Aquest manual descriu les funcionalitats principals. Aquí pots afegir una introducció general.

\section{Instal·lació}
Pas a pas per a la instal·lació:
\begin{itemize}
    \item Descarregar el programari des de la pàgina oficial.
    \item Executar l'instal·lador.
    \item Seguir les instruccions de configuració.
\end{itemize}

\section{Ús Bàsic}
Llistat de funcionalitats bàsiques:
\begin{enumerate}
    \item Iniciar el programa des del menú principal.
    \item Crear un nou projecte.
    \item Desar el treball.
\end{enumerate}

\section{Resolució de Problemes}
Algunes solucions per a problemes comuns:
\begin{itemize}
    \item Si no es pot iniciar: verificar la instal·lació.
    \item Si hi ha errors: consultar els registres del sistema.
    \item Contactar amb suport tècnic si el problema persisteix.
\end{itemize}

\section{Conclusió}
Aquest manual proporciona els passos bàsics. Si necessites més informació, consulta la documentació oficial.

\end{document}
